\section*{Synopsis}
\addcontentsline{toc}{section}{Synopsis}

This master thesis describes the work of the student Pierre Ecarlat during his 6-months internship at National Institute of Informatics, Tokyo. The goal of this project was to do real-time segmentation on aerial views (drone-derived data) using Deep Learning methods. These methods had to provide a segmentation as accurate and quick as possible, considering that we may want to use it in real-time, on a drone. Regarding the literature, it seems that DL methods applied to aerial views are quite rare, and there are no works on real-time semantic segmentation. Indeed, it is a challenging project considering the diversity of the views (fields, cities, ...), the similarity of the features on the images (not a huge difference between someone and a car), and the constraints of the drone (light GPU). 

We will present our research works on this master thesis, from the state of the art to our results and conclusion. First, we will present the Deep Learning methods focusing on the classification and segmentation problems in computer vision. We will also compare some available architectures, and present our own one, which is our main research contribution. Then, we will present the problem of the learning dataset we will have to give to our architecture, and also point out the problem of their diversity, introducing the necessity of transfer learning. Our project is finally divided into five main sections that will be presented in the third chapter, just after defining our methodology and the tools we will use. An impartial overview of our results will, then, be presented, and will be followed by a discussion section, where we will have a brief analysis on what should be improved in this project, and what we brought to the literature with our work.

\vspace{1cm}

\noindent
\textit{Keywords :} Deep Learning, Pixel-wise segmentation, Aerial Views, Deep Transfer Learning, Neural Networks Compression
