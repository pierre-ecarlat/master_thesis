\chapter*{Introduction}
\addcontentsline{toc}{section}{Introduction}


Deep Learning (DL) has been proven these last decades as a powerful recognition method by its success in recent computer vision competitions and its efficiency on recent innovative technologies. DL methods have many applications, including speech recognition (voice control), complex models structures learning (learning how to play a complex game) or, most especially, image processing for computer vision (autonomous cars, scene understanding, motion analysis, object tracking...). Some of these applications requires to do semantic segmentation (pixel-wise labelling) on images, and this involves to learn images features from a large dataset.

The goal of this project was to find how to perform efficiently real-time semantic segmentation on aerial-views, comparing different DL architectures. Understanding how does these architectures work and considering their efficiency should then let us to know which one is the most appropriate in the context of aerial-views. The contribution, in a research point of view, was to find some ways to improve the accuracy of our model by using multiple novel methods or by modifying directly our model. To handle these changes and methods implementation, we also wrote a deep learning framework facilitating the use of different way of learning.

This master thesis will present, first, how does the DL methods for computer vision work while introducing the semantic segmentation problem and presenting different existing networks. It will also present the characteristics of a dataset, and what we expect to use for this project. Then, the methodology of our experiments will be described, followed by the results and, finally, a discussion concluding on the efficiencies of the different tested methods.




